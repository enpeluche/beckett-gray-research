\documentclass[12pt]{beamer}
\usepackage[T1]{fontenc}
\usepackage[utf8]{inputenc}
\usepackage{lmodern}
\usetheme{Warsaw}
\usepackage[francais]{babel}
\usepackage{amsthm, amsmath, amssymb}
\usepackage{tikz-cd}
\usepackage{pgfgantt}
\usepackage{caption}
\usepackage[natbib=true,style=authoryear,backend=bibtex,useprefix=true]{biblatex}

\usepackage{stmaryrd}


\addbibresource{references/bibliography.bib}

\title{Exploration des Codes de Gray et des Codes de Beckett-Gray}
\author{Lucas Noirot et Mattéo Mennesson}
\institute{Université de Montpellier}
\date{Année universitaire 2023-2024}
\titlegraphic{\includegraphics[width=0.2\textwidth]{images/Gray.jpg}\hspace{2cm}\includegraphics[width=0.2\textwidth]{images/Beckett.jpg}}


\newtheorem{proposition}{Proposition}
\newtheorem{what}{Définition}
\newtheorem{ex}{Exemple}
\begin{document}


%---------------------------------------------------------------------------------------------%

\begin{frame}
    \titlepage
\end{frame}

%---------------------------------------------------------------------------------------------%

\begin{frame}
    \frametitle{Organisation du Travail}
    \centering
    \resizebox{!}{200}{
        \begin{tikzpicture}
            \centering
            \begin{ganttchart}[
                    vgrid, hgrid = {draw = none, draw = none},
                    canvas/.append style = {draw = none},
                    title/.style = {fill = none},
                    milestone label font = \tiny,
                    group label font = \tiny,
                    title label font = \tiny \footnotesize,
                    bar label node/.style = {text width = 4cm, align = right, font = \scriptsize\RaggedLeft, anchor = east},
                    milestone label node/.style = {text width = 3cm, align = right, font = \scriptsize\RaggedLeft, anchor = east},
                    group label node/.style = {text width = 4cm, align = right, font = \scriptsize\RaggedLeft, anchor = east},
                    progress label text = {}
                ]{1}{16},

                \gantttitle{Numéro de la semaine}{16} \\
                \gantttitlelist{1,...,16}{1} \\

                \ganttbar[bar/.append style={fill=blue!30}]{Lecture du survey de Torsten Mütze}{1}{3} \\
                \ganttbar[bar/.append style={fill=red!30}]{Codage code de Gray réfléchis en JavaScript}{3}{5} \\
                \ganttbar[bar/.append style={fill=green!30}]{Rapport, partie de l'article de Mütze}{5}{9} \\
                \ganttbar[bar/.append style={fill=yellow!30}]{Lecture des articles sur le codes de Beckett-Gray}{5}{7} \\
                \ganttbar[bar/.append style={fill=purple!30}]{Code génération de code de Beckett-Gray}{7}{14} \\
                \ganttbar[bar/.append style={fill=orange!30}]{Rapport, partie Beckett-Gray}{10}{14} \\
                \ganttbar[bar/.append style={fill=pink!30}]{Exploitation des résultats du code de génération}{13}{16} \\

            \end{ganttchart}
        \end{tikzpicture}
    }

\end{frame}

%---------------------------------------------------------------------------------------------%

\begin{frame}
    \frametitle{Les codes de Gray}
    \centering
    000 → 001→ 011→ 010→ 110→ 111→ 101→ 100
    \begin{table}
        \includegraphics[width=0.1\textwidth]{images/n=3.png}
        \caption*{Code de Gray $n=3$}
    \end{table}
    Minimiser le nombre de bits qui changent.
\end{frame}

%---------------------------------------------------------------------------------------------%

\begin{frame}
    \begin{what}
        On dit que deux cycles hamiltonien $c_1=(v_1, \cdots, v_n)$ et $c_2=(w_1, \cdots, w_n)$ sont isomorphes si il existe $\sigma \in \mathcal{G}_n$ tel que :

        Pour tout $i \in \{1, \cdots, n\}$, $v_i=\sigma(w_i)$ où $\sigma (w_i)$ désigne la permutation des coordonnées de $w_i$.
    \end{what}

    \begin{ex} Soit $\sigma = (12)$
        \sigma (\textcolor{red}{1}\textcolor{green}{0}0  \rightarrow \textcolor{red}{1}\textcolor{green}{0}1 \rightarrow\textcolor{red}{1}\textcolor{green}{1}1\rightarrow \textcolor{red}{1}\textcolor{green}{1}0\rightarrow\textcolor{red}{0}\textcolor{green}{1}0\rightarrow \textcolor{red}{0}\textcolor{green}{1}1\rightarrow \textcolor{red}{0}\textcolor{green}{0}1\rightarrow \textcolor{red}{0}\textcolor{green}{0}0) =\textcolor{green}{0}\textcolor{red}{1}0  \rightarrow \textcolor{green}{0}\textcolor{red}{1}1 \rightarrow\textcolor{green}{1}\textcolor{red}{1}1\rightarrow \textcolor{green}{1}\textcolor{red}{1}0\rightarrow\textcolor{green}{1}\textcolor{red}{0}0\rightarrow \textcolor{green}{1}\textcolor{red}{0}1\rightarrow \textcolor{green}{0}\textcolor{red}{0}1\rightarrow \textcolor{green}{0}\textcolor{red}{0}0
    \end{ex}
\end{frame}

%---------------------------------------------------------------------------------------------%

\begin{frame}
    \begin{what}
        La réversion d'un cycle hamiltonien $c:=v_1, \cdots v_k$ est $rev(c):=v_k, \cdots, v_1$.
    \end{what}
    \begin{ex}
        $rev(000 \rightarrow 001\rightarrow 011\rightarrow 010\rightarrow 110\rightarrow 111\rightarrow 101\rightarrow 100) = 100 \rightarrow101 \rightarrow111\rightarrow 110\rightarrow 010\rightarrow 011\rightarrow 001\rightarrow 000$
    \end{ex}

\end{frame}

%---------------------------------------------------------------------------------------------%

\begin{frame}
    \frametitle{Graphe de retournement et hypercube}
    \begin{table}
        \includegraphics[width=1\textwidth]{images/hypercubes.png}
        \caption*{Cycle hamiltonien en rouge}
    \end{table}
\end{frame}

%---------------------------------------------------------------------------------------------%

\begin{frame}
    \frametitle{Les codes de Beckett-Gray}
    Un bit à 1 peut passer à 0 seulement si c'est celui qui est passé à 1 il y a le plus longtemps.
    \begin{table}

        \includegraphics[width=0.06\textwidth]{images/n=5 red.png}
        \caption*{\centering Représentation d'un code de Beckett-Gray pour n=5, où le bit à 1 depuis le plus de temps est en rouge}
    \end{table}
\end{frame}

%---------------------------------------------------------------------------------------------%

\begin{frame}
    On remarque que les intervalles ne peuvent jamais être emboîtés. \newline
    \begin{table}

        \begin{tabular}{cc}
            \includegraphics[width=0.1\textwidth]{images/n=4 bon.png}
            \hspace{2cm}
            \includegraphics[width=0.1\textwidth]{images/n=4 pas bon.png} \\
        \end{tabular}
        \caption*{Code de Beckett-Gray partiel pour $n=4$, correct et incorrect}
    \end{table}
    \begin{proposition}
        La réversion d'un code de Beckett-Gray est un code de Beckett-Gray.
    \end{proposition}

    \begin{proposition}
        Un isomorphisme d'un code de Beckett-Gray est un code de Beckett-Gray.
    \end{proposition}
\end{frame}

%---------------------------------------------------------------------------------------------%

\begin{frame}
    \frametitle{Heuristiques}

    \underline{Question ouverte} : est-ce qu'il existe un code de Beckett-Gray pour n=9 ?\newline

    Pour $n = 5$, il faut 64 476 166 appels
    de fonctions pour générer tous les codes de Beckett-Gray et un temps d’exécution de 2.839 secondes en moyenne.\newline

    Comment réduire le temps d'exécution?\newline


    $\Rightarrow$ Implémentation d'heuristiques pour réduire la taille de l'arbre de recherche.
\end{frame}

%---------------------------------------------------------------------------------------------%

%parler ici de Gp
\begin{frame}
    \frametitle{Heuristique : sommets pendants}

    \begin{what}Un sommet pendant est un sommet ayant un degré de $1$.
    \end{what}

    \begin{table}

        \includegraphics[width=0.5\textwidth]{images/gdrggeg.png}
        \caption*{\centering Chemin simple pour n=4}
    \end{table}
\end{frame}

%---------------------------------------------------------------------------------------------%

\begin{frame}
    Il y a deux sommets pendant : pas possible
    \frametitle{Heuristique : sommets pendants}
    \begin{table}

        \includegraphics[width=0.5\textwidth]{images/BG_7_exemple.png}
        \caption*{\centering Graphe réduit avec les sommets restants}
    \end{table}
\end{frame}

%---------------------------------------------------------------------------------------------%

\begin{frame}
    \frametitle{Heuristique : propriété eulérienne}
    \begin{definition}
        Un pont d'un graphe connexe est une arête dont la suppression rend le graphe non connexe.
    \end{definition}

    \begin{table}

        \includegraphics[width=0.6\textwidth]{images/GHam.png}
        \caption*{\centering Cycle hamiltonien pour un code de Beckett-Gray n=5}
    \end{table}
\end{frame}

%---------------------------------------------------------------------------------------------%

\begin{frame}
    \frametitle{Heuristique : Nombre d’itération pour changer un 1 en 0}

    \begin{table}
        \renewcommand{\arraystretch}{1.2} % Adjust row height
        \setlength{\tabcolsep}{2pt}
        \begin{tabular}{|c|c|c|c|c|c|c|c|c|c|}
            \hline
            Nombre de 1 consécutifs & 1 & 2   & 3    & 4    & 5     & 6    & 7    & 8 \\
            \hline
            Nombre d'occurrences    & 0 & 720 & 1320 & 4920 & 11040 & 8400 & 2400 & 0 \\
            \hline
        \end{tabular}
    \end{table}

    \begin{ex}

        \textcolor{red}{1}00 →\textcolor{red}{1}01 →\textcolor{red}{1}\textcolor{green}{1}1→ \textcolor{red}{1}\textcolor{green}{1}0→ 0\textcolor{green}{1}0→ 0\textcolor{green}{1}1→ 001→ 000 \newline

        en \textcolor{red}{rouge} 4 bit à 1 consécutifs \newline

        en \textcolor{green}{vert} 4 bit 1 consécutifs

    \end{ex}
    Problème $\rightarrow$ on ne peut pas savoir à l'avance le nombre maximum de 1 consécutifs.
\end{frame}

%---------------------------------------------------------------------------------------------%

\begin{frame}
    \frametitle{Résultats sur la génération des codes de Beckett-Gray}

    Génération des codes de Beckett Gray n = 5 en 14.5 secondes en Python.\newline

    Génération des codes de Beckett Gray n = 5 en 0.428 secondes et  6 654 406 appels de fonction en C++, 6.6 fois plus rapide que sans heuristique.

    \begin{table}
        \includegraphics[scale=0.25]{images/NbAppel.png}
        \caption*{Nombre d'appels de fonction pour chaque profondeur à $n=5$}
    \end{table}
\end{frame}

%---------------------------------------------------------------------------------------------%

\begin{frame}
    \frametitle{Résultats sur la génération des codes de Beckett-Gray}
    \begin{table}
        \centering

        \begin{tabular}{cc}
            \includegraphics[width=0.4\textwidth]{images/BG7.png}
            \includegraphics[width=0.4\textwidth]{images/n=8.png} \\
        \end{tabular}
        \caption*{Code de Beckett-Gray $n=7$ et Code de Beckett-Gray partiel pour $n=8$}
    \end{table}
\end{frame}

%---------------------------------------------------------------------------------------------%

\begin{frame}
    \frametitle{Résultats sur la génération des codes de Beckett-Gray}

    \begin{table}
        \centering

        \begin{tabular}{cc}
            \includegraphics[width=0.5\textwidth]{images/Temps n=7.png}
            \includegraphics[width=0.5\textwidth]{images/Temps n=8.png} \\
        \end{tabular}
        \caption*{Temps d'exécution pour générer un code de Beckett-Gray partiel pour $n=7$ et $n=8$}
    \end{table}
\end{frame}

%---------------------------------------------------------------------------------------------%

\begin{frame}
    \frametitle{Études statistiques}

    \underline{But} : déduire des propriétés plus globales qui nous permettraient d’améliorer les temps de génération de listes exhaustives. \newline

    \begin{proposition}
        On trouve $1920$ codes de Beckett-Gray cycliques pour n = 5, on les numérote par ordre de génération : $c_1, c_2, c_3, \cdots$
    \end{proposition}

    \begin{proposition}
        On dénombre 8 codes de Beckett-Gray non-isomophes.
    \end{proposition}

\end{frame}

%---------------------------------------------------------------------------------------------%

\begin{frame}

    On va travailler avec la représentation décimales des codes, par exemple :
    $100 \rightarrow101 \rightarrow111\rightarrow 110\rightarrow 010\rightarrow 011\rightarrow 001\rightarrow 000$ devient
    $4 \rightarrow5 \rightarrow7\rightarrow 6\rightarrow 2\rightarrow 3\rightarrow 1\rightarrow 0$

    \frametitle{Tables de transition}

    \begin{what}
        Une table de transition est une matrice $n \times n$ où $n=2^k$ est le nombre de valeurs du code de Beckett-Gray sur k bits.

        Chaque cellule $(i, j)$ de la matrice représente la probabilité que l’entier $i$ soit suivi par l’entier $j$ dans l'ensemble des codes de Beckett-Gray pour $k$ fixé.
    \end{what}
\end{frame}

%---------------------------------------------------------------------------------------------%

\begin{frame}

    \begin{table}
        \begin{tabular}{ccc}
            \includegraphics[width=0.5\textwidth]{images/transition_BGC_5.png}
            \includegraphics[width=0.5\textwidth]{images/transition_BGC_6.png}
            \\
        \end{tabular}
        \caption*{Tables de transition pour $k=5$ et $k=6$}
    \end{table}
\end{frame}

%---------------------------------------------------------------------------------------------%

\begin{frame}
    \frametitle{Indices des réversions}
    \begin{what}
        Soit $x, y \in \llbracket 1, 1920 \rrbracket$. Un point $(x,y)$ est bleu si le code $c_x$ est la réversion du code $c_y$.
    \end{what}

    \begin{table}
        \includegraphics[width=0.75\textwidth]{images/woooo.png}
        \caption*{Indices des codes de Beckett-Gray et de leur réversion pour $n=5$}
    \end{table}
\end{frame}

%---------------------------------------------------------------------------------------------%

%reafaire les graphiques pour qu'ils soient carrés et pas rectangulaire

\begin{frame}
    \frametitle{\'Etudes des indices des isomorphismes}


    \begin{what}
        Soit $x, y \in \llbracket 1, 960 \rrbracket$. Un point $(x,y)$ est bleu si les codes $c_x$ et $c_y$ sont isomomorphes.
    \end{what}

    \begin{table}
        \includegraphics[width=0.65\textwidth]{images/wouah.png}
        \caption*{Code de Beckett-Gray $n=5$ associés à leurs isomorphismes}
    \end{table}
\end{frame}

%---------------------------------------------------------------------------------------------%

\begin{frame}
    \frametitle{Conclusion et perspectives}

    $\rightarrow$ S'intéresser à d'autres problèmes ouverts \newline

    $\rightarrow$ Ce qu'on a mis en oeuvre \newline

    $\rightarrow$ Utiliser plus de puissance de calcul et plus de temps. \newline

    $\rightarrow$ Utiliser les statistiques pour développer de nouvelles heuristiques
\end{frame}

%---------------------------------------------------------------------------------------------%

\begin{frame}
    \centering
    Merci pour votre attention
\end{frame}

%---------------------------------------------------------------------------------------------%

\end{document}