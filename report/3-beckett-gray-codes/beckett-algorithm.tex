
\subsection{Algorithme}
Pour écrire un algorithme qui va générer des codes de Beckett-Gray, on commence avec l'algorithme suivant écrit en
pseudo-code que l'on va légèrement modifier par la suite.\\

La fonction GC est une implémentation de génération des codes de Gray. Elle fonctionne de manière récursive en itérant
sur chaque position possible du mot binaire pour inverser un bit dans la séquence en cours de génération. Lorsqu'une
inversion est possible, elle est effectuée, puis la fonction est appelée de manière récursive pour générer les autres
codes de Gray. Lorsque la longueur de la séquence atteint $2^n$, la récursion s'arrête et le code de Gray est affiché.\\

Cette approche récursive permet de générer de manière exhaustive tous les codes de Gray possibles.\newline

\begin{pseudo}
    procedure GC(s, x, maxpos :integer)
    local i
    if s >= 2^n then
    Print()
    else
    for i = 0 to Min(n-1, maxpos) do
    x = Flip(x, i)
    if avail[x] then
    avail[x] = false
    bgc[s] = x
    GC(s + 1, x, Max(maxpos, i))
    avail[x] = true
    x = Flip(x, i)
\end{pseudo}

\begin{itemize}
    \item \textbf{bgc} : un tableau global pour stocker la liste des codes Gray sur des chaînes binaires de longueur $n$.
    \item \textbf{avail} : un tableau booléen global pour suivre quelles chaînes sont encore disponibles.
    \item \textbf{x} : la valeur entière de la chaîne actuellement à la fin de la liste.
    \item \textbf{s} : la longueur de la liste partielle.
    \item \textbf{maxpos} : la position de bit la plus élevée qui a été définie à $1$ à un moment donné.
    \item \textbf{Flip(x, i)} : une fonction qui retourne la valeur entière obtenue en inversant le $i$-ème bit dans la
          représentation binaire de $x$.
    \item \textbf{Print()} : une fonction qui imprime le code de Gray bgc.
\end{itemize}

Pour initialiser l'algorithme, nous fixons $avail[i] = true$ pour $i \in\{1, \cdots, n-1\}$, et $avail[0] = false$,
fixons $bgc[0] = 0$, puis appelons $GC(1, 0, 0)$. \\

L'algorithme générera tous les codes Gray sur des chaînes binaires de longueur $n$. Pour générer uniquement les codes Gray
cycliques sur des chaînes binaires de longueur $n$, la fonction Print est appelée uniquement si la dernière chaîne dans
la liste est une puissance de $2$ (elle diffère d'un bit de $0$).
