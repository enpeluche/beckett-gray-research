
\subsection{Études des isomorphismes}
Une chose à remarquer en regardant les indices des $8$ codes non isomorphes générés, ceux-ci se trouvent tous parmi
les premiers codes générés. Cela pourrait indiquer une généralisation possible : l'algorithme pourrait être interrompu
assez tôt, les derniers codes pouvant être obtenus par isomorphisme. \\

Il est pertinent d'étudier l'indice d'un code généré par rapport à l'indice de ses codes isomorphes , et de représenter ces
indices sur un graphique, comme illustré ci-dessous. Un point (x,y) est dans ce graphique si le code $c_1$ d'indice $x$ est
isomorphe au code $c_2$ d'indice $y$.\\

Il est à noter que ce tableau s'est fait avec les $960$ codes à qui on a enlevé leur réversion, le graphique peut être altéré.
\begin{table}[h!]
    \centering
    \caption{Temps d'exécution pour générer un code de Beckett-Gray partiel}
    \begin{tabular}{cc}
        \includegraphics[width=0.75\textwidth]{wouah.png} \\
    \end{tabular}
\end{table}
\FloatBarrier

On remarque encore une fois tout de suite la caractère remarquable de cette figure, dans les $96$ premiers codes générés, $8$
codes non isomorphes sont présent, nous n'avons aucune idée de pourquoi cette valeur de $96 = 1920/20$, mais on pourrait penser
que les \num{94841} codes de Beckett-Gray non isomorphes pour $n=6$ soient générés assez rapidement, en testant sur les \num{1000000}
premiers codes on trouve environs \num{20000} codes non isomorphes. On pourrait alors créer une heuristique qui coupe très rapidement
l'algorithme, évitant de générer tous les codes, pouvant être obtenus par isomorphisme, puis par réversion, bien entendu, tout cela
ne reste que des conjectures.

\begin{itemize}
    \item \underline{Motifs en Damier} : Le graphique présente un motif en damier distinct, avec des carrés de points bleus alternant
          avec des zones vides. Cela indique que les isomorphismes se produisent régulièrement entre certains groupes de codes. Si on
          considère le nombre de zones bleus, on trouve 10 carrés bleu en ordonné et 15 en abscisse.
    \item \underline{Distribution des Points} : Les points sont densément regroupés en blocs. Chaque bloc représente un ensemble de
          codes de Beckett-Gray qui sont mutuellement isomorphes. Ces blocs sont séparés par des espaces vides, suggérant que les isomorphismes
          sont contenus au sein de groupes spécifiques de codes et ne se produisent pas entre ces groupes.
    \item \underline{Taille des Blocs} : Les blocs ont une taille variable, mais ils apparaissent généralement sous forme de carrés ou
          de rectangles réguliers. Cela indique une structure sous-jacente dans la manière dont les codes de Beckett-Gray sont isomorphes les
          uns aux autres.
    \item \underline{Interprétation des Blocs} : Les blocs représentant les isomorphismes montrent que pour un certain indice de code
          de Beckett-Gray, il existe un ensemble défini d'autres indices avec lesquels il est isomorphe. Ces ensembles semblent être bien
          délimités et non aléatoires.
    \item \underline{Concentration de points} : Vers la fin du graphique, il semble y avoir une densité accrue de rectangles.
\end{itemize}

En conclusion le graphique démontre que les isomorphismes entre les codes de Beckett-Gray ne sont pas distribués de manière aléatoire
mais suivent une structure régulière et répétitive. Les motifs en damier et les blocs de points bleus révèlent que les isomorphismes se
produisent fréquemment au sein de groupes spécifiques de codes.\\
