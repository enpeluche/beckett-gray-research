

\subsection{Fréquence des nombres par colonne}

Quand on examine la fréquence des nombres par colonne, on obtient ces graphiques. On prend par exemple le $5$-ème élément de
tous les codes de Beckett-Gray générés pour n=5, et on regarde la fréquence de chaque nombre qui apparaît.

\begin{table}[h!]
    \centering
    \caption{Histogrammes des colonnes $5$, $6$, $19$ et $28$ des codes de Beckett-Gray.}
    \begin{tabular}{cc}
        \includegraphics[scale=0.4]{hist_5.png}
        \includegraphics[scale=0.4]{hist_6.png} \\
        \includegraphics[scale=0.4]{hist_19.png}
        \includegraphics[scale=0.4]{hist_28.png}
    \end{tabular}
\end{table}
\FloatBarrier

Ce qui est intéressant, c'est de constater qu'il existe toujours des classes d'équiprobabilité dans ces colonnes, et qu'il y a
peu de probabilités différentes d'apparaître dans ces colonnes. Il y a aussi peu d'éléments différents qui apparaissent dans
chaque colonne. Une notion plus pertinente à étudier serait celle des tables de transitions.
