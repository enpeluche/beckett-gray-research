
\subsection{Conclusion}
De cette analyse, nous concluons que les codes de Beckett-Gray présentent une étonnante symétrie, et qu'il serait intéressant
d'exploiter ces propriétés pour développer de nouvelles heuristiques ou des algorithmes probabilistes. Ces statistiques
suggèrent l'existence potentielle de codes de Beckett-Gray pour des entiers supérieurs à $8$. Étant donné que tous les
codes de cette partie ont été réalisés en Python, il serait judicieux de les transposer en C\texttt{++} et d'utiliser davantage
de jeux de données pour $n=6$. De plus, nous pourrions générer d'autres codes pour $n=7$, car nous n'en disposons actuellement
que d'un seul. Il serait également intéressant d'étudier si les premiers codes de Beckett-Gray générés sont non isomorphes pour
$n=6$ et $n=7$. En observant ces structures, il est difficile de déterminer si les codes de Beckett-Gray pour $n=6$ contiennent
des "sous-codes" de Beckett-Gray pour $n=5$.

\newpage