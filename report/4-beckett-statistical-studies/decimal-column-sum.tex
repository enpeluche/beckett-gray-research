
\subsection{Somme des colonnes de la représentation décimale des codes de Beckett-Gray}

Les deux tableaux suivant présentent la somme des colonnes des valeurs décimales des codes de Beckett-Gray pour $n=5$ et
$n=6$, respectivement. Il est à noter que le deuxième tableau est incomplet, car seuls \num{1856008} codes ont été générés
au moment de la création du graphique sur un total \num{136 571 040}. Cette visualisation pourra être améliorée à l'avenir
avec une génération continue de codes. On observe un motif symétrique sur les deux graphiques, illustrant l'équivalence sous
réversion des codes de Beckett-Gray. De plus, on remarque deux 'pics' aux extrémités des graphiques, où la somme des colonnes
croît initialement puis diminue, créant une sorte de plateau entre les deux. Ce caractère très symétrique et régulier semble
être créé par les isomorphismes des codes de Beckett-Gray.

\begin{table}[h!]
    \centering
    \caption{Somme des colonnes de la représentation décimale des codes de Beckett-Gray pour $n = 5$ et $n= 6$}
    \begin{tabular}{cc}
        \includegraphics[width=0.45\textwidth]{sum_col_BGC_5.png}
        \includegraphics[width=0.45\textwidth]{sum_col_BGC_6.png} \\
    \end{tabular}
\end{table}
\FloatBarrier