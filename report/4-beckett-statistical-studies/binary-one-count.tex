

\subsection{Nombre de bit à $1$ dans la représentation binaire des codes de Beckett-Gray}
Voici les premier tableaux représentant le nombre de bit à $1$ dans la représentation binaire de chaque élément des premiers
codes de Beckett-Gray générés pour $n = 5$, $n = 6$ et $n = 7$.

\begin{table}[h!]
    \centering
    \caption{Nombre de bit à $1$ dans la représentation binaire des codes de Beckett-Gray pour $n = 5$}
    \begin{tabular}{cc}
        \includegraphics[width=0.5\textwidth]{BGC_5_nb_ones_line_0.png}
        \includegraphics[width=0.5\textwidth]{BGC_5_nb_ones_line_2.png} \\
    \end{tabular}
\end{table}
\FloatBarrier

\begin{table}[h!]
    \centering
    \caption{Nombre de bit à 1 dans la représentation binaire des codes de Beckett-Gray pour $n = 6$}
    \begin{tabular}{cc}
        \includegraphics[width=0.5\textwidth]{BGC_6_nb_ones_line_0.png}
        \includegraphics[width=0.5\textwidth]{BGC_6_nb_ones_line_2.png} \\
    \end{tabular}
\end{table}
\FloatBarrier


\begin{table}[h!]
    \centering
    \caption{Nombre de bit à $1$ dans la représentation binaire des codes de Beckett-Gray pour $n = 7$}
    \begin{tabular}{c}
        \includegraphics[width=0.5\textwidth]{BGC_7_nb_ones_line_0.png}
    \end{tabular}
\end{table}
\FloatBarrier

Pour un $n$ donné, on observe de légères fluctuations entre les motifs des codes de Beckett-Gray successifs. On remarque
principalement des variations entre deux graphe qui ressemblent à des permutations de segments de graphe, ainsi que des
transitions où un bit à $1$ est remplacé par un bit à $0$ et vice versa, ce qui se traduit par des sommets devenant des
vallées et des vallées devenant des sommets. Ces fluctuations se manifestent sous forme de dents de scie sur le graphique,
où l'axe des abscisses représente le numéro de la colonne d'un code de Beckett-Gray et l'axe des ordonnées le nombre de
$1$ dans sa représentation binaire. On remarque que deux codes de Beckett-Gray isomorphes ont le même nombre de bit à
$1$ dans leur représentation binaire, et donc le même graphique, on s'attend ainsi à avoir seulement $8$ graphiques vraiment
différents pour $n=5$.
