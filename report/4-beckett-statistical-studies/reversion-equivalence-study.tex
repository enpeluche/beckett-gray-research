

\subsection{Étude de l'équivalence sous réversion}
Il est pertinent d'étudier l'indice d'un code généré par rapport à l'indice de sa réversion, et de représenter ces indices sur
un graphique, comme illustré ci-dessous. Un point (x,y) est dans ce graphique si le code $c_1$ d'indice $x$ est la réversion du
code $c_2$ d'indice $y$.

\begin{table}[h!]
    \centering
    \caption{Indices des codes de Beckett-Gray et de leur réversion pour $n=5$}
    \begin{tabular}{cc}
        \includegraphics[width=0.75\textwidth]{woooo.png} \\
    \end{tabular}
\end{table}
\FloatBarrier

Cette figure est remarquable, on remarque plusieurs choses d'un seul coup d'oeil :
\begin{itemize}
    \item \underline{Distribution des Points} : Les points sont répartis de manière relativement uniforme, mais il existe des zones
          de plus forte densité. Certaines zones présentent une structure plus claire, ce qui peut indiquer des motifs récurrents dans les
          indices des codes inversés.
    \item \underline{Zones de Concentration} : On observe des bandes horizontales et verticales où les points sont plus concentrés,
          suggérant que certains indices de codes inversés apparaissent fréquemment en relation avec des indices spécifiques de codes
          originaux.
    \item \underline{Symétrie et Répartition} : Il y a une symétrie au niveau des structures apparente autour de la diagonale principale
          (qui va de (0,0) à (1920,1920)).
    \item \underline{Gaps et espaces vides} : Il y a des zones vides où aucun point n'est présent. On l'interprète comme ceci : parmi
          les $384 = 1920/5$ premiers codes aucune réversion n'est présente dans les $384$ premiers, et ainsi de suite. Ces gaps peuvent être
          dus à la nature des transformations de Beckett-Gray et de leurs inversions, qui ne couvrent pas tous les indices de manière uniforme.
          Ces gaps sont au nombre de $5$, cela laisse suggérer que pour $n=6$ ces gaps seraient au nombre de $6$. Ces gaps interviennent tous
          les $384 = 1920/5$ codes, cela laisse suggérer que ces gaps interviendraient tous les $\num{22761840} = \num{136571040}/6$ codes. A
          l'heure actuelle de l'écriture de ce paragraphe, Lucas a généré $5$ millions de codes de Beckett-Gray pour $n=6$, ne pouvant pour
          l'instant pas vérifier la présence d'un gap ou non, mais en calculant, on ne trouve aucune réversion, ce qui laisse suggérer que
          cette hypothèse est vraie. On remarque que les $384$ derniers codes sont donc inutiles à générer, comme ils sont des réversions des
          premiers, on pourrait donc couper l'algorithme $384$ codes avant la fin, et cela laisse suggérer qu'on pourrait couper \num{22761840}
          codes avant la fin pour $n=6$, ce qui peut être une énorme amélioration.
\end{itemize}

En conclusion le graphique montre que les indices des codes de Beckett-Gray et leurs reversions ont une relation complexe. Les motifs de
concentration et les zones vides suggèrent que certaines inversions sont plus courantes que d'autres, et que l'inversion n'est pas une
transformation simple des indices.
