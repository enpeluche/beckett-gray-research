
\subsection{Codes de Beckett-Gray non-isomorphes}

On s'est intéressé à générer les codes de Beckett-Gray non isomorphes pour $n=5$ et $n=6$. Pour cela, nous avons d'abord
identifié les codes équivalents sous réversion, puis en utilisant les permutations, trouvés les codes non isomorphes dans
le cas $n=5$.\\

\begin{proposition}
    On dénombre $8$ codes de Beckett-Gray cycliques non isomorphes pour $n=5$.
\end{proposition}

\begin{preuve}
    Grâce aux classes d'équivalence et à la bijection $rev$, on peut décomposer le nombre de codes de Beckett-Gray pour $n=5$
    de la façon suivante :
    $$1920 = 2 \times 5! \times 8$$
\end{preuve}

\begin{proposition}
    On dénombre \num{94841} codes de Beckett-Gray cycliques non isomorphes pour $n=6$.
\end{proposition}

\begin{preuve}
    Grâce aux classes d'équivalence et à la bijection $rev$, on peut décomposer le nombre de codes de Beckett-Gray pour $n=6$
    de la façon suivante :
    $$\num{136571040} = 2 \times 6! \times \num{94841}$$
\end{preuve}

Voici une représentation des $8$ codes de Beckett-Gray sur une figure en forme de disque :

\begin{table}[h!]
    \centering
    \caption{Représentation graphique des $8$ codes de Beckett-Gray non isomorphes}
    \vspace{0.2cm}
    \begin{tabular}{cc}
        \includegraphics[width=0.2\textwidth]{iso1.png}
        \includegraphics[width=0.2\textwidth]{iso2.png}
        \includegraphics[width=0.2\textwidth]{iso3.png}
        \includegraphics[width=0.2\textwidth]{iso4.png} \\
        \includegraphics[width=0.2\textwidth]{iso5.png}
        \includegraphics[width=0.2\textwidth]{iso6.png}
        \includegraphics[width=0.2\textwidth]{iso7.png}
        \includegraphics[width=0.2\textwidth]{iso8.png}
    \end{tabular}
\end{table}
\FloatBarrier