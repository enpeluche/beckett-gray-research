
\subsection{Cycles isomorphes dans l'hypercube vu comme un graphe de retournement}

Une autre notion que nous allons examiner sur l'hypercube $\mathcal{Q}_n$ en le considérant comme un graphe de retournement,
concerne les cycles isomorphes. La notion de cycles isomorphes nous amène à la notion de code de Gray isomorphes, c'est donc
une notion très utile pour étudier la structure des codes de Gray. Dans toute la suite on suppose $n > 3$. On définit
préalablement ce qu'on entends par l'égalité de deux cycles. Deux cycles $c_1$ et $c_2$ sont dis égaux si ils parcourent les
mêmes sommets dans le même ordre.

\begin{definition}
    Une permutation de l'ensemble $\{1, \cdots, n\}$ est une bijection de $\{1, \cdots, n\}$ dans $\{1, \cdots, n\}$.
\end{definition}

\begin{definition}
    On appelle groupe symétrique d’indice $n$, noté $\mathcal{G}_n$ l’ensemble des permutations de l’ensemble $\{1, 2, 3, \cdots , n\}$.
\end{definition}

\begin{definition}
    Une permutation circulaire $\sigma$ de $\{a_1, \cdots, a_k\}$ est définie par $\sigma(a_j) = a_{j+1}$ pour tout
    $j \in \{1, \cdots,  n-1\}$ et $\sigma(i_n) = \sigma (a_1)$
\end{definition}

Du point de vue informatique, on prend le dernier élément d'une liste, qu'on rajoute au début de la liste. L'inverse d'une permutation
circulaire consiste à prendre le premier élément de la liste et le mettre à la fin de cette liste. C'est une involution.

\begin{proposition}
    On a $|\mathcal{G}_n|=n!$.
\end{proposition}

\begin{definition}
    On dit que deux codes de Gray $c_1=(v_1, \cdots, v_n)$ et $c_2=(w_1, \cdots, w_n)$ sont isomorphes si il existe
    $\sigma \in \mathcal{G}_n$ tel que :

    Pour tout $i \in \{1, \cdots, n\}$, $v_i=\sigma(w_i)$ où $\sigma (w_i)$ désigne la permutation des coordonnées de $w_i$.
\end{definition}

On note $c_1 \simeq c_2$ si $c_1$ est isomorphe à $c_2$.

\begin{proposition}
    $\simeq$ est une relation d'équivalence.
\end{proposition}

\begin{preuve}
    Il faut montrer que $\simeq$ est réflexive, symétrique et transitive.
    \begin{itemize}
        \item $\simeq$ est \underline{réflexive} : $$c_1 \simeq c_1$$ en revenant à la définition de cycles isomorphes et
              prenant la permutation identité.
        \item $\simeq$ est \underline{symétrique} : si $c_1 \simeq c_2$ alors $c_2 \simeq c_1$ en revenant à la définition
              de cycles isomorphes et prenant la permutation inverse.
        \item $\simeq$ est \underline{transitive} : si $c_1 \simeq c_2$ et $c_2 \simeq c_3$ alors $c_1 \simeq c_3$ en revenant
              à la définition de cycles isomorphes et en prenant la composition des deux permutations.
    \end{itemize}
\end{preuve}

On peut donc parler de classes d'équivalences de cycles isomorphes que l'on notera $eq_c := \{c ~|~ c \simeq a\}$, le choix
d'un représentant $c$ est arbitraire.

\begin{proposition}
    Dans $\mathcal{Q}_n$, $\# eq_c = n!$.
\end{proposition}

Si on trouve un cycle hamiltonien sur $\mathcal{Q}_n$ on peut en trouver $n!$ autres cycles hamiltonien. Cela nous donne une
borne inférieur du nombre de code de cycles hamiltonien dans $\mathcal{Q}_n$.

\begin{proposition}
    Il y a $\binom{n}{i}$ mots possibles de longueur $n$ sur l'alphabet $\{0,1\}$ qui possède $i$ fois la lettre $1$.
\end{proposition}

On retrouve la formule $2^n= \binom{n}{i} + \cdots + \binom{n}{n}$.

\begin{proposition}
    Une permutation d'un mot binaire qui possède $i$ bits à $1$ possède toujours $i$ bits à $1$.
\end{proposition}

On a donc une façon de détecter si deux cycles ne sont pas isomorphes en comparant le nombre de bit à $1$ dans leur éléments successifs.

\begin{proposition}
    Il n'y a que deux mots de longueur $n$ qui sont fixés par toutes les permutations de $\mathcal{G}_n$ : le mot $1^n$ et le mot $0^n$.
\end{proposition}

On peut l'interpréter comme le fait que certains sommets du graphe sont une étape incontournable pour tout isomorphisme d'un code de Gray.
