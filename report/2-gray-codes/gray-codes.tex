
\section{Les codes de Gray.}

Le codage de Gray tient son nom de l'ingénieur américain Frank Gray. Plusieurs codages similaires aux codes de Gray existent,
avec des objectifs différents. L'un des objectifs premiers est de générer facilement et rapidement tous les objets d'un ensemble
dans un ordre bien précis. Le concept plus généralement connu comme code de Gray combinatoire n'exige plus de coder les nombres
mais de lister différents objets d'un ensemble, de sorte que deux objets consécutifs dans la liste diffèrent seulement d'un
"petit" changement. Ainsi le codage de Gray est un cas particulier de codage de Gray combinatoire. En informatique, les codes de
Gray sont utiles pour minimiser le risque d'erreur pendant la transmission d'une information par exemple, étant donné que pour
deux valeurs consécutives, il suffit de changer la valeur d'un seul bit.\newline

Ce code est par exemple utile pour des capteurs de positions absolue, par exemple sur des règles optiques. En effet, si on utilise
le code binaire pur, pendant le passage de la position cinq $(101)_2$ à six $(110)_2$ (changement simultané de $2$ bits) il y a un
risque de passage transitoire par quatre $(100)_2$ ou sept $(111)_2$, ce que le code Gray évite. Dans un codage de Gray cyclique,
le passage du maximum (quinze sur $4$ bits) à zéro se fait également en ne modifiant qu'un seul bit. Ceci permet par exemple
d'encoder un angle sur une machine automatique ou un robot.\newline

\begin{definition}
    Un code de Gray est un codage des entiers telle que lorsque on augmente l'entier d'une unité, le codage correspondant ne change
    que d'un seul bit. Pour le codage binaire tel que l'on le connaît, cette propriété n'est pas respectée, par exemple pour passer
    de $1$ à $2$, on change deux bits dans leurs représentation binaire, afin de passer de $001$ à $010$.
\end{definition}

\begin{definition}
    Un code de Gray est cyclique quand son premier élément et son dernier élément diffèrent d'un seul bit.
\end{definition}

Le tableau suivant illustre ce qu'est un code de Gray, pour un codage cyclique sur $3$ bits.

    {
        \centering
        \begin{tabular}{|c|c|c|} \hline
            Nombre & Codage binaire & Codage de Gray \\ \hline
            $1$    & $001$          & $000$          \\ \hline
            $2$    & $010$          & $001$          \\ \hline
            $3$    & $011$          & $011$          \\ \hline
            $4$    & $100$          & $010$          \\ \hline
            $5$    & $101$          & $110$          \\ \hline
            $6$    & $110$          & $111$          \\ \hline
            $7$    & $111$          & $101$          \\ \hline
        \end{tabular} \par}
