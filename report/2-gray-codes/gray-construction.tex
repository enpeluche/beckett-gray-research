
\subsection{Construction}
On peut construire récursivement un code de Gray de la manière suivante, cette construction est aussi appelée code binaire réfléchi.

$$L_0 = \epsilon$$
$$L_n = 0L_{n-1}, 1\overline{L_{n-1}}$$

où on a $\Sigma = \{0, 1\}$, $\epsilon$ représente le mot vide $aL_n$ représente la concaténation entre $a$ et $L_n$,
avec $a \in \{0, 1\}$ et $\overline{L}$ la réversion de $L$ (le premier élément de L devient le dernier élément de $L$,
le deuxième élément de $L$ devient l'avant dernier élément de $L$, etc $\cdots$). Par exemple $\overline{abcd} = dcba$.
On remarque le parallèle entre cette construction récursive et la construction récursive dans les hypercubes.\\


Par exemple, voici les premières valeurs des listes $L_1$, $L_2$ et $L_3$.

$$L_1 = 0,1$$
$$L_2 = 00, 01, 11, 10$$
$$L_3 = 000, 001, 011, 010, 110, 111, 101, 100$$
Ainsi, un codage de Gray suivant cette méthode récursive existe toujours.
