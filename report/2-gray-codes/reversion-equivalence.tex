
\subsection{Équivalence sous réversion}

\begin{definition}
    La réversion d'un cycle hamiltonien $c:=v_1, \cdots v_k$ est $rev(c):=v_k, \cdots, v_1$.
\end{definition}

\begin{proposition}
    Si $c$ est un cycle hamiltonien, $rev(c)$ est un cycle hamiltonien.
\end{proposition}

$rev$ est une bijection de l'ensemble des cycles hamiltoniens, on note $c_1 \simeq_r c_2$ si $c_1 = rev(c_2)$.
On notera que $rev$ est une involution.

\begin{proposition} $\simeq_r$ n'est pas une relation d'équivalence, mais elle possède des propriétés intéressantes :
    \begin{itemize}
        \item si $c_1 \simeq_r c_2$ alors $c_2 \simeq_r c_1$.
        \item $c \not\simeq c$.
        \item si $c_1 \simeq_r c_2$ et $c_2 \simeq_r c_3$ alors $c_1 = c_3$.
    \end{itemize}
\end{proposition}

\begin{preuve}
    La preuve est un jeu de manipulation de la notation $rev$
    \begin{itemize}
        \item si $c_1 \simeq_r c_2$ alors $c_1 = rev(c_2)$ et donc $rev(c_1) = rev(rev(c_2))$ et donc $rev(c_1) = c_2$
              finalement $c_2 \simeq_r c_1$.
        \item $c \neq rev(c)$ dès que le cycles est de longueur strictement supérieur à $3$ et que les sommets sont tous distincts.
        \item si $c_1 \simeq_r c_2$ et $c_2 \simeq_r c_3$ alors $c_1 = rev(c_2)$ et $c_2 = rev(c_3)$ alors $c_1 = rev(rev(c_3))$ et $c_1=c_3$
    \end{itemize}
\end{preuve}

La fonction de réversion est une bijection, on identifie $c$ et sa réversion $rev(c)$ si ils sont la réversion l'un de
l'autre et on note $rev_c = \{c, rev(c)\}$.

\begin{proposition}
    Pour tout cycle hamiltonien : $\# rev_c = 2$.
\end{proposition}

Si on trouve un cycle hamiltonien sur $\mathcal{Q}_n$ on peut trouver $2 \times n!$ autres cycles hamiltonien. Cela nous donne
une borne inférieur du nombre de code de cycles hamiltonien dans $\mathcal{Q}_n$.

Avec cette correspondance entre les cycles et les codes de Gray, on obtient la notion d'isomorphisme de code de Gray et de réversion
de code de Gray.
