
\section{Organisation du travail}

Nous avons commencé à travailler sur le TER à la fin du mois de janvier, initialement en groupe de quatre personnes.
Notre première tâche a été de lire et de nous approprier l'article de Torsten Mütze \cite{CGC}. Ensuite, nous avons
développé des codes en JavaScript pour générer des codes de Gray de différents types. Le choix de JavaScript s'explique
par la possibilité de visualiser ces codes sur une page web. Par la suite, il nous a été demandé de choisir un problème
non résolu de l'article. Nous avons opté pour celui concernant les codes de Beckett-Gray et commencé notre rapport en \LaTeX.
Nous nous sommes documentés avec \cite{BGC_fast} et \cite{BGC}. \\

Vers la sixième semaine de travail, deux membres ont quitté notre groupe, nous laissant à deux pour nous répartir les
tâches. À ce moment-là, Lucas s'est principalement concentré sur la rédaction du rapport tandis que Mattéo s'occupait
du codage. Une problématique de notre TER était de développer des heuristiques. Pour le codage, nous avons abandonné
JavaScript au profit de Python, un langage que nous maîtrisons mieux. Cependant, une fois les codes fonctionnels obtenus,
nous avons décidé de les transcrire en C\texttt{++} pour améliorer les temps d'exécution, délaissant également Python
pour ces tâches, bien qu'il reste utilisé pour l'exploitation des résultats. \\

Après cela, Lucas a principalement effectué des analyses statistiques sur les codes de Beckett-Gray pour $n=5$ et
$n=6$ afin de voir si l'on pouvait en déduire des propriétés, tandis que Mattéo a poursuivi l'écriture du rapport.
Voici le diagramme de Gantt associé :

\resizebox{!}{200}{
    \begin{tikzpicture}
        \centering
        \begin{ganttchart}[
                vgrid, hgrid = {draw = none, draw = none},
                canvas/.append style = {draw = none},
                title/.style = {fill = none},
                milestone label font = \tiny,
                group label font = \tiny,
                title label font = \tiny \footnotesize,
                bar label node/.style = {text width = 4cm, align = right, font = \scriptsize\RaggedLeft, anchor = east},
                milestone label node/.style = {text width = 3cm, align = right, font = \scriptsize\RaggedLeft, anchor = east},
                group label node/.style = {text width = 4cm, align = right, font = \scriptsize\RaggedLeft, anchor = east},
                progress label text = {}
            ]{1}{16},

            \gantttitle{Numéro de la semaine}{16} \\
            \gantttitlelist{1,...,16}{1} \\

            \ganttbar[bar/.append style={fill=blue!30}]{Lecture du survey de Torsten Mütze}{1}{3} \\
            \ganttbar[bar/.append style={fill=red!30}]{Codage code de Gray réfléchis en JavaScript}{3}{5} \\
            \ganttbar[bar/.append style={fill=green!30}]{Rapport, partie de l'article de Mütze}{5}{9} \\
            \ganttbar[bar/.append style={fill=yellow!30}]{Lecture des articles sur le codes de Beckett Gray}{5}{7} \\
            \ganttbar[bar/.append style={fill=purple!30}]{Code génération de code de Beckett Gray}{7}{14} \\
            \ganttbar[bar/.append style={fill=orange!30}]{Rapport, partie Beckett Gray}{10}{14} \\
            \ganttbar[bar/.append style={fill=pink!30}]{Exploitation des résultats du code de génération}{13}{16} \\

        \end{ganttchart}
    \end{tikzpicture}
}
\newpage